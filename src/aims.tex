\section{Motivation \& Aims}
  % TODO: stop referring to Zcash, cite Kachina
  Currently Bitcoin~\cite{bitcoin} transactions store permanently the
  pseudonymous addresses of transacting parties in the clear on the blockchain.
  As shown by~\cite{DBLP:conf/fc/AndroulakiKRSC13}, correlating this information
  with social network graphs to deanonymize parties is practical and relatively
  cheap. Avoiding the reuse of addresses does not protect from such attacks
  against privacy. Techniques proposed in the past such as
  CoinJoin~\cite{DBLP:conf/trustcom/MaurerNF17} need active user coordination,
  are prone to DoS attacks and provide only heuristic privacy guarantees that
  can be violated by a determined adversary.

  Existing work~\cite{DBLP:conf/sp/Ben-SassonCG0MTV14,zcash-protocol,kachina}
  shows that it is possible and practical to integrate zero-knowledge proof
  systems into blockchains. The aim of this proposal is to enable the use of
  zk-SNARKs in Bitcoin. Such an extension could bring a number of useful
  features to Bitcoin as follows: %, which is currently the cryptocurrency with the highest
%  market capitalization\footnote{\url{https://coinmarketcap.com}}.
  \begin{itemize}
    \item the potential for improved privacy guarantees,
    \item a meaningful extension of the bitcoin scripting language in a sandboxed manner,
    \item a more expressive and privacy-friendly layer-1 system that will facilitate more powerful layer-2 applications. For instance, protocols such as~\cite{DBLP:conf/sss/DeckerW15,lightning,perun,DBLP:conf/systor/LindNEKPS18,sprites}
    for nearly instant and low-fee smart contract capabilities in a trustless
    manner.
  \end{itemize}
  The above can be achieved with minimal changes to Bitcoin: the new feature
  will be available through four new Bitcoin
  Script\footnote{\url{https://en.bitcoin.it/wiki/Script}} opcodes that can be
  introduced through a soft fork~\cite{DBLP:conf/fc/ZamyatinSJSWK18}.
