\section{Motivation \& Aims}
  Currently Bitcoin~\cite{bitcoin} transactions store permanently the
  pseudonymous addresses of transacting parties in the clear on the blockchain.
  As shown by~\cite{DBLP:conf/fc/AndroulakiKRSC13}, correlating this
  information with social network graphs to deanonymize parties is practical and
  relatively cheap.  Avoiding the reuse of addresses does not protect from such
  attacks against privacy.  Techniques proposed in the past such as
  CoinJoin~\cite{DBLP:conf/trustcom/MaurerNF17} need active user coordination,
  are prone to DoS attacks and provide only heuristic privacy guarantees that
  can be violated by a determined adversary.

  Zcash~\cite{DBLP:conf/sp/Ben-SassonCG0MTV14,zcash-protocol} is another
  blockchain with semantics similar to Bitcoin. The main difference is that the
  former attempts to solve the issue of privacy leakage by employing the use of
  zk-SNARKs~\cite{DBLP:conf/stoc/BitanskyCCT13,DBLP:conf/eurocrypt/Groth16} for
  users that wish to use them. At a high level, each such transaction carries a
  zero-knowledge proof of the fact that it transfers coins between some parties.
  This proof ensures that no new coins are created out of thin air, but does not
  disclose neither the value of coins nor the addresses of the implicated
  parties, creating thus one big anonymity set for all parties that have ever
  transacted using zcash ``shielded''\footnote{we borrow the terms
  ``transparent'' and ``shielded'' from Zcash.} addresses.

  The aim of this proposal is to show that it is possible and practical to
  integrate similar privacy capabilities in Bitcoin through a soft
  fork~\cite{DBLP:conf/fc/ZamyatinSJSWK18}. Such an extension could bring
  stronger privacy to Bitcoin, which is the cryptocurrency with the highest
  market capitalization\footnote{\url{https://coinmarketcap.com}}.
