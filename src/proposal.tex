\section{Proposal}
  Existing zk-SNARK systems need a \emph{structured reference string} (SRS) to
  generate and verify proofs. This string is public information, but its
  generation must be carried out by an honest party: if the Adversary is the one
  that generates the SRS, then it can use information from this generation
  procedure to later create valid proofs for false statements, completely
  subverting the zk-SNARK security.

  One way to alleviate this problem is to make the SRS \emph{updateable}, i.e.
  to allow the SRS to change throughout the lifetime of the zk-SNARK system.
  Each update would be based on the previous in a manner that ensures that even
  if a single updater has been honest, then no one can generate valid proofs to
  false statements. Such a capability is provided by
  SONIC~\cite{DBLP:conf/ccs/MallerBKM19}, we therefore choose this as our
  zk-SNARK system.

  We propose to add four new opcodes to the Bitcoin Script:
  \texttt{OP\_SRS\_CREATE}, \texttt{OP\_SRS\_UPDATE}, \texttt{OP\_SNARK\_CREATE}
  and \texttt{OP\_SNARK\_UPDATE}. To simplify the description and avoid a number
  of complications, we require that if one of these opcodes appears in a script,
  it has to be the only opcode in that script, otherwise the transaction is
  invalid. Further investigation is needed to determine whether it is possible
  to lift this limitation.

  \texttt{OP\_SRS\_CREATE} generates a new SRS. It can be followed by either two
  or three data fields. The first data field contains the new SRS, the second
  the proof of its correctness and the optional third field contains (a
  reference to) the description of an NP relation $R$ that all SNARKs under this
  SRS must observe. If present, the last field defines a state machine within
  which the SNARK system will operate.

  \texttt{OP\_SRS\_UPDATE} updates an existing SRS. It must be followed by
  exactly three data fields. The first data field contains the
  outpoint\footnote{reference to a specific transaction output already in the
  ledger} that carries the previous SRS, the second contains the new SRS and the
  third the correctness proof of the update.

  \texttt{OP\_SNARK\_CREATE} moves some normal bitcoins to a private context. It
  must be followed by three or four data fields: An outpoint that references the
  SRS, (a reference to) the description of an NP relation $R$ that all
  subsequent SNARKS that use these coins must obey -- needed only if the
  relevant \texttt{OP\_SRS\_CREATE} did not specify such a relation -- an
  initial statement $x$ and a zk-SNARK that proves that this is a valid
  statement (i.e., $\exists w: R(x, w) = 1$). Similarly to
  \texttt{OP\_SRS\_CREATE}, the relation defines the semantics of the state
  machine to be used by subsequent SNARKs.

  \texttt{OP\_SNARK\_UPDATE} updates the state of the private state machine with
  a new valid statement. Following Bitcoin semantics, it must consume the
  outputs of the transactions that contain the previous valid statements $x_1,
  \dots, x_n$. It must be followed by exactly three data fields: An outpoint
  that references the SRS, a statement $x'$ and a zk-SNARK that proves that the
  previous statements together with the new statement obey the overarching
  relation (i.e., $\exists w: R(x_1 \wedge \dots \wedge x_n \wedge x', w) = 1$).
  The relation $R$ can be designed so that, under particular circumstances,
  e.g., when the contract is resolved, $x'$ can be a Bitcoin output that can be
  consumed by the current Bitcoin Script, therefore converting the private coins
  back to normal bitcoins.

  We note that the SNARK relation can be specified either at the SRS creation or
  during the conversion of bitcoins to private coins. In the former case coins
  that are converted in separate transactions can freely interact, but the
  relation is tied with the specific SRS (or any of its updates) and can never
  change. In the second case on the other hand, a single SRS can cater to
  various relations, but coins converted in various transactions cannot
  interact.

  A potential specific application of the above system (with the relation
  defined together with the SRS) is the reproduction of
  Zcash~\cite{DBLP:conf/sp/Ben-SassonCG0MTV14,zcash-protocol} capabilities. We
  can allow transactions that provide the same privacy guarantees as
  ``shielded'' Zcash transactions do. Another example application in which the
  relation can be defined together with the SRS is a private version of the
  lightning network~\cite{lightning}. An example usecase of defining the
  relation at conversion is private smart contracts, e.g. gambling or joint
  savings accounts.

  As a precaution against malicious SRSs, full nodes should furthermore keep
  track of the number of coins paid into and withdrawn from a particular SRS. In
  order to be valid, the transaction containing the zk-SNARK must withdraw at
  most as much coins as the ones remaining in the used SRS. This is a simple
  safeguard that ensures the firewall property~\cite{DBLP:conf/sp/GaziKZ19},
  i.e. that no bitcoins can be generated out of thin air.

  Taproot\footnote{\url{https://github.com/sipa/bips/blob/bip-schnorr/bip-taproot.mediawiki}}
  defines a mechanism for specifying semantics for new opcodes, which we can use
  to introduce the   new opcodes. In case some of the above requirements
  cannot be enforced through this upgrade mechanism (e.g. the requirement of a
  single zk-SNARK opcode per script), we can leverage other update paths, such
  as increasing the SegWit version or using the ``annex'' field that Taproot
  provides.

  As this is a soft fork, full nodes with old software will accept all
  transactions described above as valid. They will also accept transactions with
  fake zk-SNARKs, so every node is advised to update. In order for these rules
  to be enforced, a supermajority of the mining power should have such
  capability activated. An update strategy similar to that used for enabling
  SegWit\footnote{\url{https://github.com/bitcoin/bips/blob/master/bip-0141.mediawiki}}
  can be employed.
